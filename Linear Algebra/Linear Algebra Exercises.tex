\documentclass[../main]{subfiles}
\begin{document}
\chapter{Linear Algebra}
\section{Eigenvalue and Eigenvector}
\begin{bbox}{The eigenvalues of $A^2$}
    If $A$ has eigenvalues $\lambda_i$. Then the eigenvalues of $A^2$ are $\lambda_i^2$.
    \begin{proof}
        Well, my first intuition is to think about the diagonalization of $A$ and the result becomes clear.
        \newline
        A rigorous proof is also not hard: 
        \begin{enumerate}
            \item \begin{align*}
                A\vec v = \lambda \vec v &\implies A^2 \vec v = A \lambda\vec v = \lambda^2 \vec v.
            \end{align*}
        \item 
            The algebraic multiplicity of the eigenvalues $\lambda^2_i$ of $A^2$ is the same as the eigenvalues $\lambda_i$ of $A$:
            \[
            \det (A^2 - \lambda^2 I) = \det (A+\lambda I)\det (A-\lambda I).
            \]
            This means that
        \end{enumerate}
    \end{proof}
\end{bbox}
\section{Determinant}
\subsection{The Determinant is the Product of Eigenvalues}
\begin{proof}
    Let $A$ be a matrix with eigenvalues $\lambda_i$. The key idea of the proof uses the characteristic polynomial.
    \begin{enumerate}
        \item Consider the characteristic polynomial
        \[
        p(\lambda) = |\lambda I - A| = c_0 + c_1 \lambda + \dots + \lambda^n
        \]
        Note that the characteristic polynomial is monic.
        \item We can obtain $c_0$ by
        \[
        p(0) = c_0 = |0\cdot I - A| = (-1)^n \det A
        \]
        \item Note that the eigenvalues $\lambda_i$ are roots of the characteristic polynomial so 
        \[
        p(0) = \prod_i(0-\lambda_i) = (-1)^n \lambda_i
        \]
        \item Lastly, 
        \[
        c_0 = (-1)^n \prod_{i}\lambda_i  = (-1)^n \det A
        \]
        so 
        \[
        \det A = \prod_i \lambda_i
        \]
    \end{enumerate}
\end{proof}
\section{Trace}
\subsection{Trace is Equal to the Sum of Eigenvalues}
\begin{bbox}{Trace is Equal to the Sum of Eigenvalues}
    Let $A$ be an $n\times n$ matrix with eigenvalues $\lambda_i$. Show that 
    \[
    \Tr(A) = \sum_i \lambda_i
    \]
\end{bbox}
\begin{solution}
    \begin{enumerate}
        \item The proof is similar to that of "the determinant is product of eigenvalues," i.e. we work with the characteristic polynomial. \textcolor{red}{TO BE FILLED IN} 
    \end{enumerate}
\end{solution}
\subsection{An Inequality relating Trace and Determinant}
\begin{bbox}{{2018 Summer Practice Problem, \# 18}}
Suppose $\Sigma$ is a non-negative definite matrix of $n\times n$ real entries and real eigenvalues. Show that 
\[
\Tr(\Sigma^2) \geq n \cdot \det(\Sigma)^{2/n}.
\]
\end{bbox}
\begin{solution}
    \begin{enumerate}
        \item Let $\{\lambda_i\}$ be the eigenvalues of $\Sigma$. To make some progress, let's write the trace as 
        \[
        \Tr(\Sigma) = \sum_i \lambda_i^2        
        \]
        \item By the Arithmetic Mean - Geometric Mean inequality,
        \[
        \frac{\sum_{i=1}^n \lambda_i^2}{n} \geq \sqrt[n]{\prod_{i=1}^n \lambda_i^2} \implies \Tr(\Sigma^2) \geq n \det(\Sigma)^{\frac{2}{n}}
        \]
    \end{enumerate}
\end{solution}
\section{Core Competency Exam Questions}
\begin{bbox}{{2020 September Exam, \#8}}
    For every $n\geq 1$, let $A_n$ be an $n\times n$ symmetric matrix with non-negative entries. Let $R_n(i):= \sum_{j=1}^n A_n(i,j)$ denote the ith row/column sum of $A_n$. Assume that 
    \[
    \lim_{n\to \infty} \max_{1\leq i \leq n} |R_n(i) - 1| = 0.
    \]
    Let $\lambda_n \geq 0$ denote an eigenvalue with the largest absolute value, and let $\vec x = (x_1, \dots, x_n)$ denote its corresponding eigenvector. 
    \begin{itemize}
        \item Show that 
        \[
        \frac{1}{n} \sum_{i,j=1}^n A_n(i,j) \to 1
        \]
        \item Show that $\lambda_n |x_i| \leq \max_{1\leq j \leq n} |x_j| R_n(i)$.
        \item Using parts one and two, show that 
        \[
        \lambda_n \to 1.
        \]
    \end{itemize}
\end{bbox}
\begin{solution}
    For the first part, let's just write something down:
    \begin{enumerate}
        \item 
        \begin{align*}
        \frac{1}{n} \sum_{i,j=1}^n A_n(i,j) &= \frac{1}{n} \sum_{i=1}^n R_n(i)
    \end{align*}
        \item \begin{align*}
        \left|\frac{1}{n} \sum_{i,j=1}^n A_n(i,j) - 1\right| &= \left|\frac{1}{n} \sum_{i=1}^n R_n(i) - 1\right|\\
        &\leq \max_{1\leq i\leq n} |R_n(i) - 1| \to 0
    \end{align*}
    \end{enumerate}
    
    For the second part, 
    \begin{enumerate}
        \item By assumption, \begin{align*}
        A_n \vec x &= \lambda_n \vec x, \quad \lambda_n x_i = \sum_{j=1}^n A_n(i,j)x_j\\
        \lambda_n |x_i|&\leq \sum_{j=1}^n A_n(i,j)|x_j| = R_n(i) \max_{1\leq j \leq n} |x_j|.
    \end{align*}
    \end{enumerate}
    
    For the third part, we first use the Rayleigh quotient. For any nonzero vector $v\in\mathbb R^n$,
    \begin{align*}
      \lambda_n \;=\;\max_{\|u\|_2=1}u^T A_n\,u
      \;\ge\;
      \max_{\|u\|_2=1} \sum_{i,j=1}^n A_n(i,j)u_i u_j\\
      \;\geq\;\sum_{i,j=1}^n A_n(i,j)\frac{1}{\sqrt{n}}\frac{1}{\sqrt{n}}
      \;=\;
      \frac1n\sum_{i,j}A_n(i,j) \to 1.
    \end{align*}
    For the other direction, we use part two. Choose $k$ such that $|x_k| = \max_{j} |x_j|$
    \begin{align*}
        \lambda_n \leq &\frac{x_k}{x_k} R_n(k) \to 1
    \end{align*}
\end{solution}
\begin{bbox}{(Straightforward) {2021 May Exam, \#7}}
    Suppose that $A=(a_{ij})_{1\le i,j \le 2}$ is a $2\times 2$ symmetric matrix, with $a_{11} = a_{22} =\frac{3}{4}$ and $a_{12} = a_{21} = \frac{1}{4}$. 
    \begin{itemize}
        \item Find the eigenvalues and eigenvectors of the matrix $A$.
        \item Compute $\lim_{n\to +\infty} a_{12}^{(n)}$ where $a_{i,j}^{(n)}$ denotes the $ij$th entry of the matrix $A^n$.
    \end{itemize}
\end{bbox}
\begin{solution}
    The first part is standard. Set up the characteristic polynomial and solve for its roots:
    \[
    p(\lambda) = \det(A-\lambda I) = 0 \implies \lambda = \frac{1}{2}, 1
    \]
    The eigenvector corresponding to $\lambda=1$ is $\begin{bmatrix}
         1/\sqrt{2}  \\
         1/\sqrt{2} 
    \end{bmatrix}$. The eigenvector corresponding to $\lambda=\frac{1}{2}$ is $\begin{bmatrix}
        1/\sqrt{2}\\ -1/\sqrt{2}
    \end{bmatrix}$.
    \newline
    For the second part. We should use diagonalization; otherwise, matrix exponential would be hard to compute.
    \[
    A = PDP^{-1}
    \]
    where $D$ is the diagonal matrix whose diagonal entries are the eigenvalues. $P^{-1}$ is the matrix whose the columns are the corresponding eigenvectors. So $D = \begin{bmatrix}
        1 & 0\\
        0 & \frac{1}{2}
    \end{bmatrix}$ and $ P^{-1} = \begin{bmatrix}
        1 & 1\\
        1 & -1
    \end{bmatrix}, \quad P= \begin{bmatrix}
        \frac{1}{2} & \frac{1}{2}\\
        \frac{1}{2} & -\frac{1}{2}
    \end{bmatrix}$. 
    \begin{align*}
    A^n &= P \begin{bmatrix}
        1^n & 0\\
        0 & \frac{1}{2}^n
    \end{bmatrix} P^{-1} = \begin{bmatrix}
        \frac{1}{2} & \frac{1}{2}\\
        \frac{1}{2} & -\frac{1}{2}
    \end{bmatrix} \begin{bmatrix}
        1 & 0\\
        0 & 0.5^n
    \end{bmatrix} \begin{bmatrix}
        1 & 1\\
        1 & -1
    \end{bmatrix}\\
    &= \begin{bmatrix}
        \frac{1}{2} & \frac{1}{2}\\
        \frac{1}{2} & -\frac{1}{2}
    \end{bmatrix} \begin{bmatrix}
        1 & 1\\
        0.5^n & -(0.5^n)
    \end{bmatrix}\\
    a_{12}^n = \frac{1}{2} - \frac{1}{2} \cdot (-(0.5^n)) \to \frac{1}{2}.
    \end{align*}
    This question is straightforward in my opinion!
\end{solution}
\begin{bbox}{{2021 Sept Exam, \#6}}
Let $A \in \mathbb R^{m\times n}$ be an $m\times n$ matrix with $n < m$. Suppose that $\lambda_1,\lambda_2,\dots, \lambda_n$ and $\vec v_1,\dots, \vec v_n$ are the eigenvalues and eigenvectors of $A^TA$. What can we say about ALL the eigenvalues and eigenvectors of $AA^T$. Justify your answer.
\end{bbox}
\begin{solution}
    When it comes $AA^T$, especially when $A$ is non-symmetric or even non-square, we should think of Singular Value Decomposition SVD! Let $A = U \Sigma V^{-1}$ be its SVD. Then $A^T = V \Sigma^{T} U^{-1}$.
    The singular values of $A$ are the square root of the eigenvalues of $AA^T$, and we see that $A$ and $A^T$ share the same singular values. Note that $U$ is composed of orthonormal eigenvectors of $AA^T$ and $V$ is composed of orthonormal eigenvectors of $A^TA$. $AA^T \vec v_i = \lambda \vec v_i $
\end{solution}
\begin{bbox}{Eigenvalue of Orthogonal Matrix}
    Let $A$ be a $3\times 3$ real-valued matrix such that $A^T A = AA^T = I_3$ and $\det (A) = 1$. Prove that $1$ is an eigenvalue of $A$.
\end{bbox}
\begin{solution}
    % Well, our first intuition should be that $1$ must a root of the characteristic polynomial. The problem wants to tell us that $A$ is orthogonal. The problem also told us something about the determinant, which kinda means that we should consider the characteristic polynomial. Okay, consider 
    % \[
    % \det (A - I) = 0 \iff \det(A^T(A-I))=0 \iff \det(I - A^T) =0
    % \]
    Since the problem wants to tell us that $A$ is orthogonal, we should be thinking of the length-preserving property. Let $\lambda$ be an eigenvalue of $A$ and $\vec v$ be a corresponding unit eigenvector. Then 
    \[
    \|A\vec v\| = \sqrt{\vec v^T A^T A \vec v} = 1 =\|\lambda \vec v\| = |\lambda|
    \]. 
    \newline
    The determinant is the product of the eigenvalues and $-1$ cannot be the only eigenvalue of $A$ because $(-1)^3 = -1 \neq 1 = \det A$.
\end{solution}
\begin{bbox}{(Straightforward) Trace of the square of a symmetric matrix is zero means zero matrix}
    Let $A$ be an $n\times n$ symmetric matrix such that $\Tr(A^2) = 0$. Show that $A = 0_{n\times n}$.
    \newline
    Hint: Use the fact that $\Tr(ABC) = \Tr(CAB)$.
\end{bbox}
\begin{solution}
    The hint apparently wants us to apply the spectral theorem to obtain a diagonalization $A = Q \Lambda Q^T$.
    \[
    \Tr A^2 = \Tr \left(Q\Lambda^2 Q^T\right) = \Tr\left(Q^T Q\Lambda^2\right) =\Tr(\Lambda^2)= 0.
    \]
    The trace is equal to the sum of the eigenvalues (to be honest, with this fact, we don't really need the hint), i.e. the diagonal of $\Lambda^2$ is zero. Since the entries of $\Lambda^2$ are non-zero, $\Lambda^2 = 0$ and hence $\Lambda = 0$. Therefore $A=0$.
\end{solution}

\begin{bbox}{Eigenvectors are the same iff Multiplication commutes}
    Let $A,B\in \mathbb R^{n\times n}$ have respective eigendecompositions $Q_1 D_1Q_1^T$ and $Q_2 D_2 Q_2^T$ (recall this means each $D_i$ is a diagonal matrix of eigenvalues and each $Q_i$ is an orthogonal matrix). Prove that $Q_1 = Q_2$ if and only if $AB = BA$. You may assume that $A, B$ do not have any repeated eigenvalues.
\end{bbox}
\begin{solution}
    Suppose $AB = BA$, consider an eigenpair $\lambda$ and $\vec v$ of $A$. 
    \[
    BA \vec v = \lambda B\vec v. = AB\vec v.
    \]
    This means that $B\vec v$ is an eigenvector of $A$ corresponding to the eigenvalue $\lambda$. This then imply to $A$ and $B$ share the same set of eigenvalues $\lambda_i$ with corresponding eigenvectors $\vec v_i$ and $B\vec v_i$. For $Q_1 = Q_2$, we need to show that $\vec v_i \propto B\vec v$: 
    \[
    A B\vec v = \lambda B\vec v, \implies B\vec v \propto \vec v
    \] since the eigenspaces of $A$ are all one-dimensional.
    \newline
    The other direction is easier. Suppose $Q_1 = Q_2$, then 
    \[
    AB = Q_1 D_1 Q_1^T Q_2 D_2 Q_2^T = Q_2 D_2 Q_2^TQ_1 D_1 Q_1^T = BA
    \]
\end{solution}
\begin{bbox}{(Straightforward) Eigenvalue of $uv^T$}
    Let $A = uv^T \in \mathbb R^{n\times n}$ be a rank-one matrix, i.e. $u, v \in \mathbb R^n$. Suppose $u,v \neq 0_n$. Find, with proof, all the eigenvalues of $A$.
\end{bbox}
\begin{solution}
    Let $\lambda$ be an eigenvalue of $A$ and $\vec x$ be a corresponding eigenvector, then 
    \[
    A \vec x= uv^T \vec x = \lambda \vec x 
    \]
    Note that 
    \[ 
    u v^T x = u \langle v, x\rangle = \lambda \vec x
    \]
    This means that $\vec x$ and $\vec u$ share the same direction. So 
    \[
    A \vec u = u v^T u = \lambda u
    \]
    Therefore, 
    \[
    \lambda = \vec v^T \vec u
    \]
    There can be no other eigenvalues because $A$ has rank-one.
    \newline
    Comment: Should find this problem straightforward.

\end{solution}
\end{document}